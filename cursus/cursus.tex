\documentclass[11pt]{article}

\textwidth 16cm \textheight 23cm \evensidemargin 0cm
\oddsidemargin 0cm \topmargin -2cm
\parindent 0pt
\parskip \medskipamount


\usepackage[dutch]{babel}
\usepackage{amssymb}
\usepackage{amsmath}
\usepackage[utf8]{inputenc}
\usepackage[normalem]{ulem} % strikethrough normal text with \sout{text}
\usepackage{cancel} % strikethrough in math mode with \cancel{text}

\usepackage{subfig}
\usepackage{graphicx}

\usepackage[table]{xcolor}

\usepackage{pgf,tikz}
\usetikzlibrary{arrows}

\usepackage{color}
\newcommand{\todo}[1]{\textcolor{red}{\##1\#}}
\newcommand{\question}[1]{\textcolor{blue}{\##1\#}}

\newcommand{\degree}{\ensuremath{^\circ}}

\newtheorem{definition}{Definitie}

\begin{document}

\section{Beschrijvende Statistiek}

\subsection{Inleiding}

\subsection{Populatie en steekproef}

\subsection{Kwantitatieve en kwalitatieve kenmerken}

\subsection{Frequentie}

\subsection{Kans}

\subsection{Diagrammen}

\subsection{Gegroepeerde gegevens}

\subsection{Histogram}

\subsection{Frequentiepolygoon}

\subsection{Centrumgetallen}

\subsection{Spreidingsgetallen}

\end{document}























